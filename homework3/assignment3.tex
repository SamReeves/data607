% Options for packages loaded elsewhere
\PassOptionsToPackage{unicode}{hyperref}
\PassOptionsToPackage{hyphens}{url}
%
\documentclass[
]{article}
\usepackage{lmodern}
\usepackage{amsmath}
\usepackage{ifxetex,ifluatex}
\ifnum 0\ifxetex 1\fi\ifluatex 1\fi=0 % if pdftex
  \usepackage[T1]{fontenc}
  \usepackage[utf8]{inputenc}
  \usepackage{textcomp} % provide euro and other symbols
  \usepackage{amssymb}
\else % if luatex or xetex
  \usepackage{unicode-math}
  \defaultfontfeatures{Scale=MatchLowercase}
  \defaultfontfeatures[\rmfamily]{Ligatures=TeX,Scale=1}
\fi
% Use upquote if available, for straight quotes in verbatim environments
\IfFileExists{upquote.sty}{\usepackage{upquote}}{}
\IfFileExists{microtype.sty}{% use microtype if available
  \usepackage[]{microtype}
  \UseMicrotypeSet[protrusion]{basicmath} % disable protrusion for tt fonts
}{}
\makeatletter
\@ifundefined{KOMAClassName}{% if non-KOMA class
  \IfFileExists{parskip.sty}{%
    \usepackage{parskip}
  }{% else
    \setlength{\parindent}{0pt}
    \setlength{\parskip}{6pt plus 2pt minus 1pt}}
}{% if KOMA class
  \KOMAoptions{parskip=half}}
\makeatother
\usepackage{xcolor}
\IfFileExists{xurl.sty}{\usepackage{xurl}}{} % add URL line breaks if available
\IfFileExists{bookmark.sty}{\usepackage{bookmark}}{\usepackage{hyperref}}
\hypersetup{
  pdftitle={Week 3 Assignment},
  pdfauthor={Sam Reeves},
  hidelinks,
  pdfcreator={LaTeX via pandoc}}
\urlstyle{same} % disable monospaced font for URLs
\usepackage[margin=1in]{geometry}
\usepackage{color}
\usepackage{fancyvrb}
\newcommand{\VerbBar}{|}
\newcommand{\VERB}{\Verb[commandchars=\\\{\}]}
\DefineVerbatimEnvironment{Highlighting}{Verbatim}{commandchars=\\\{\}}
% Add ',fontsize=\small' for more characters per line
\usepackage{framed}
\definecolor{shadecolor}{RGB}{248,248,248}
\newenvironment{Shaded}{\begin{snugshade}}{\end{snugshade}}
\newcommand{\AlertTok}[1]{\textcolor[rgb]{0.94,0.16,0.16}{#1}}
\newcommand{\AnnotationTok}[1]{\textcolor[rgb]{0.56,0.35,0.01}{\textbf{\textit{#1}}}}
\newcommand{\AttributeTok}[1]{\textcolor[rgb]{0.77,0.63,0.00}{#1}}
\newcommand{\BaseNTok}[1]{\textcolor[rgb]{0.00,0.00,0.81}{#1}}
\newcommand{\BuiltInTok}[1]{#1}
\newcommand{\CharTok}[1]{\textcolor[rgb]{0.31,0.60,0.02}{#1}}
\newcommand{\CommentTok}[1]{\textcolor[rgb]{0.56,0.35,0.01}{\textit{#1}}}
\newcommand{\CommentVarTok}[1]{\textcolor[rgb]{0.56,0.35,0.01}{\textbf{\textit{#1}}}}
\newcommand{\ConstantTok}[1]{\textcolor[rgb]{0.00,0.00,0.00}{#1}}
\newcommand{\ControlFlowTok}[1]{\textcolor[rgb]{0.13,0.29,0.53}{\textbf{#1}}}
\newcommand{\DataTypeTok}[1]{\textcolor[rgb]{0.13,0.29,0.53}{#1}}
\newcommand{\DecValTok}[1]{\textcolor[rgb]{0.00,0.00,0.81}{#1}}
\newcommand{\DocumentationTok}[1]{\textcolor[rgb]{0.56,0.35,0.01}{\textbf{\textit{#1}}}}
\newcommand{\ErrorTok}[1]{\textcolor[rgb]{0.64,0.00,0.00}{\textbf{#1}}}
\newcommand{\ExtensionTok}[1]{#1}
\newcommand{\FloatTok}[1]{\textcolor[rgb]{0.00,0.00,0.81}{#1}}
\newcommand{\FunctionTok}[1]{\textcolor[rgb]{0.00,0.00,0.00}{#1}}
\newcommand{\ImportTok}[1]{#1}
\newcommand{\InformationTok}[1]{\textcolor[rgb]{0.56,0.35,0.01}{\textbf{\textit{#1}}}}
\newcommand{\KeywordTok}[1]{\textcolor[rgb]{0.13,0.29,0.53}{\textbf{#1}}}
\newcommand{\NormalTok}[1]{#1}
\newcommand{\OperatorTok}[1]{\textcolor[rgb]{0.81,0.36,0.00}{\textbf{#1}}}
\newcommand{\OtherTok}[1]{\textcolor[rgb]{0.56,0.35,0.01}{#1}}
\newcommand{\PreprocessorTok}[1]{\textcolor[rgb]{0.56,0.35,0.01}{\textit{#1}}}
\newcommand{\RegionMarkerTok}[1]{#1}
\newcommand{\SpecialCharTok}[1]{\textcolor[rgb]{0.00,0.00,0.00}{#1}}
\newcommand{\SpecialStringTok}[1]{\textcolor[rgb]{0.31,0.60,0.02}{#1}}
\newcommand{\StringTok}[1]{\textcolor[rgb]{0.31,0.60,0.02}{#1}}
\newcommand{\VariableTok}[1]{\textcolor[rgb]{0.00,0.00,0.00}{#1}}
\newcommand{\VerbatimStringTok}[1]{\textcolor[rgb]{0.31,0.60,0.02}{#1}}
\newcommand{\WarningTok}[1]{\textcolor[rgb]{0.56,0.35,0.01}{\textbf{\textit{#1}}}}
\usepackage{graphicx}
\makeatletter
\def\maxwidth{\ifdim\Gin@nat@width>\linewidth\linewidth\else\Gin@nat@width\fi}
\def\maxheight{\ifdim\Gin@nat@height>\textheight\textheight\else\Gin@nat@height\fi}
\makeatother
% Scale images if necessary, so that they will not overflow the page
% margins by default, and it is still possible to overwrite the defaults
% using explicit options in \includegraphics[width, height, ...]{}
\setkeys{Gin}{width=\maxwidth,height=\maxheight,keepaspectratio}
% Set default figure placement to htbp
\makeatletter
\def\fps@figure{htbp}
\makeatother
\setlength{\emergencystretch}{3em} % prevent overfull lines
\providecommand{\tightlist}{%
  \setlength{\itemsep}{0pt}\setlength{\parskip}{0pt}}
\setcounter{secnumdepth}{-\maxdimen} % remove section numbering
\ifluatex
  \usepackage{selnolig}  % disable illegal ligatures
\fi

\title{Week 3 Assignment}
\author{Sam Reeves}
\date{2/17/2021}

\begin{document}
\maketitle

\hypertarget{section}{%
\section{1.}\label{section}}

\hypertarget{using-the-173-majors-listed-in-fivethirtyeight.coms-college-majors-dataset-provide-code-that-identifies-the-majors-that-contain-either-data-or-statistics}{%
\subsubsection{Using the 173 majors listed in fivethirtyeight.com's
College Majors dataset, provide code that identifies the majors that
contain either ``DATA'' or
``STATISTICS''}\label{using-the-173-majors-listed-in-fivethirtyeight.coms-college-majors-dataset-provide-code-that-identifies-the-majors-that-contain-either-data-or-statistics}}

\begin{Shaded}
\begin{Highlighting}[]
\FunctionTok{library}\NormalTok{(magrittr)}
\FunctionTok{library}\NormalTok{(stringr)}
\NormalTok{url }\OtherTok{\textless{}{-}} \StringTok{"https://raw.githubusercontent.com/fivethirtyeight/data/master/college{-}majors/majors{-}list.csv"}

\NormalTok{data }\OtherTok{\textless{}{-}} \FunctionTok{data.frame}\NormalTok{(}\FunctionTok{read.csv}\NormalTok{(url))}

\NormalTok{data}\SpecialCharTok{$}\NormalTok{Major[}\FunctionTok{which}\NormalTok{(}\FunctionTok{str\_detect}\NormalTok{(data}\SpecialCharTok{$}\NormalTok{Major, }\StringTok{"DATA"}\NormalTok{))]}
\end{Highlighting}
\end{Shaded}

\begin{verbatim}
## [1] COMPUTER PROGRAMMING AND DATA PROCESSING
## 174 Levels: ACCOUNTING ACTUARIAL SCIENCE ... ZOOLOGY
\end{verbatim}

\begin{Shaded}
\begin{Highlighting}[]
\NormalTok{data}\SpecialCharTok{$}\NormalTok{Major[}\FunctionTok{which}\NormalTok{(}\FunctionTok{str\_detect}\NormalTok{(data}\SpecialCharTok{$}\NormalTok{Major, }\StringTok{"STATISTICS"}\NormalTok{))]}
\end{Highlighting}
\end{Shaded}

\begin{verbatim}
## [1] MANAGEMENT INFORMATION SYSTEMS AND STATISTICS
## [2] STATISTICS AND DECISION SCIENCE              
## 174 Levels: ACCOUNTING ACTUARIAL SCIENCE ... ZOOLOGY
\end{verbatim}

\hypertarget{write-code-that-transforms-the-data-below}{%
\section{2. Write code that transforms the data
below:}\label{write-code-that-transforms-the-data-below}}

{[}1{]} ``bell pepper'' ``bilberry'' ``blackberry'' ``blood orange''
{[}5{]} ``blueberry'' ``cantaloupe'' ``chili pepper'' ``cloudberry''\\
{[}9{]} ``elderberry'' ``lime'' ``lychee'' ``mulberry''\\
{[}13{]} ``olive'' ``salal berry''

Into a format like this:

c(``bell pepper'', ``bilberry'', ``blackberry'', ``blood orange'',
``blueberry'', ``cantaloupe'', ``chili pepper'', ``cloudberry'',
``elderberry'', ``lime'', ``lychee'', ``mulberry'', ``olive'', ``salal
berry'')

\begin{Shaded}
\begin{Highlighting}[]
\NormalTok{x }\OtherTok{\textless{}{-}} \StringTok{\textquotesingle{}[1] "bell pepper"  "bilberry"     "blackberry"   "blood orange"  }
\StringTok{[5] "blueberry"    "cantaloupe"   "chili pepper" "cloudberry"  }
\StringTok{[9] "elderberry"   "lime"         "lychee"       "mulberry"    }
\StringTok{[13] "olive"        "salal berry"\textquotesingle{}}

\NormalTok{y }\OtherTok{\textless{}{-}} \FunctionTok{str\_remove\_all}\NormalTok{(}\FunctionTok{unlist}\NormalTok{(}\FunctionTok{str\_extract\_all}\NormalTok{(x, }\StringTok{\textquotesingle{}"[a{-}z]*}\SpecialCharTok{\textbackslash{}\textbackslash{}}\StringTok{s*[a{-}z]*"\textquotesingle{}}\NormalTok{)), }\StringTok{\textquotesingle{}}\SpecialCharTok{\textbackslash{}"}\StringTok{\textquotesingle{}}\NormalTok{)}
\NormalTok{y}
\end{Highlighting}
\end{Shaded}

\begin{verbatim}
##  [1] "bell pepper"  "bilberry"     "blackberry"   "blood orange" "blueberry"   
##  [6] "cantaloupe"   "chili pepper" "cloudberry"   "elderberry"   "lime"        
## [11] "lychee"       "mulberry"     "olive"        "salal berry"
\end{verbatim}

\hypertarget{describe-in-words-what-these-expressions-will-match}{%
\section{3. Describe, in words, what these expressions will
match:}\label{describe-in-words-what-these-expressions-will-match}}

\begin{itemize}
\tightlist
\item
  (.)\textbackslash1\textbackslash1

  \begin{itemize}
  \tightlist
  \item
    Any tripled character.
  \end{itemize}
\item
  ``(.)(.)\textbackslash2\textbackslash1''

  \begin{itemize}
  \tightlist
  \item
    Any two characters followed by \textbackslash2\textbackslash1, all
    inside of quotes.
  \end{itemize}
\item
  (..)\textbackslash1

  \begin{itemize}
  \tightlist
  \item
    Any pair of characters repeated.
  \end{itemize}
\item
  ``(.).\textbackslash1.\textbackslash1''

  \begin{itemize}
  \tightlist
  \item
    Any two characters followed by \textbackslash1, any character, and
    \textbackslash1 all in quotes.
  \end{itemize}
\item
  "(.)(.)(.).*\textbackslash3\textbackslash2\textbackslash1"

  \begin{itemize}
  \tightlist
  \item
    Any character followed by any character, followed by any character,
    followed by any number of random characters, followed by
    \textbackslash3\textbackslash2\textbackslash1, all in quotes.
  \end{itemize}
\end{itemize}

\#4. Construct regular expressions to match words that:

\begin{itemize}
\tightlist
\item
  Start and end with the same character

  \begin{itemize}
  \tightlist
  \item
    \b({[}a-z{]})\w*\textbackslash1\textbackslash b
  \end{itemize}
\item
  Contain a repeated pair of letters (eg ``church'' contains ``ch''
  repeated twice.)

  \begin{itemize}
  \tightlist
  \item
    \b\w*({[}a-z{]}\{2\})\w*\textbackslash1\w*\textbackslash b
  \end{itemize}
\item
  Contain one letter repeated in at least three places (eg. ``eleven''
  contains three e's.)

  \begin{itemize}
  \tightlist
  \item
    \b\w*({[}a-z{]})\w*\textbackslash1\w*\textbackslash1\w*\textbackslash b
  \end{itemize}
\end{itemize}

\end{document}
